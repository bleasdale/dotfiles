\documentclass[10pt,a4paper]{article}

% Packages
\usepackage{fancyhdr}           % For header and footer
\usepackage{multicol}           % Allows multicols in tables
\usepackage{tabularx}           % Intelligent column widths
\usepackage{tabulary}           % Used in header and footer
\usepackage{hhline}             % Border under tables
\usepackage{graphicx}           % For images
\usepackage{xcolor}             % For hex colours
%\usepackage[utf8x]{inputenc}    % For unicode character support
\usepackage[T1]{fontenc}        % Without this we get weird character replacements
\usepackage{colortbl}           % For coloured tables
\usepackage{setspace}           % For line height
\usepackage{lastpage}           % Needed for total page number
\usepackage{seqsplit}           % Splits long words.
%\usepackage{opensans}          % Can't make this work so far. Shame. Would be lovely.
\usepackage[normalem]{ulem}     % For underlining links
% Most of the following are not required for the majority
% of cheat sheets but are needed for some symbol support.
\usepackage{amsmath}            % Symbols
\usepackage{MnSymbol}           % Symbols
\usepackage{wasysym}            % Symbols
%\usepackage[english,german,french,spanish,italian]{babel}              % Languages

% Document Info
\author{Nima (nimakarimian)}
\pdfinfo{
  /Title (c-pointers-cookbook.pdf)
  /Creator (Cheatography)
  /Author (Nima (nimakarimian))
  /Subject (C++ Pointers cookbook Cheat Sheet)
}

% Lengths and widths
\addtolength{\textwidth}{6cm}
\addtolength{\textheight}{-1cm}
\addtolength{\hoffset}{-3cm}
\addtolength{\voffset}{-2cm}
\setlength{\tabcolsep}{0.2cm} % Space between columns
\setlength{\headsep}{-12pt} % Reduce space between header and content
\setlength{\headheight}{85pt} % If less, LaTeX automatically increases it
\renewcommand{\footrulewidth}{0pt} % Remove footer line
\renewcommand{\headrulewidth}{0pt} % Remove header line
\renewcommand{\seqinsert}{\ifmmode\allowbreak\else\-\fi} % Hyphens in seqsplit
% This two commands together give roughly
% the right line height in the tables
\renewcommand{\arraystretch}{1.3}
\onehalfspacing

% Commands
\newcommand{\SetRowColor}[1]{\noalign{\gdef\RowColorName{#1}}\rowcolor{\RowColorName}} % Shortcut for row colour
\newcommand{\mymulticolumn}[3]{\multicolumn{#1}{>{\columncolor{\RowColorName}}#2}{#3}} % For coloured multi-cols
\newcolumntype{x}[1]{>{\raggedright}p{#1}} % New column types for ragged-right paragraph columns
\newcommand{\tn}{\tabularnewline} % Required as custom column type in use

% Font and Colours
\definecolor{HeadBackground}{HTML}{333333}
\definecolor{FootBackground}{HTML}{666666}
\definecolor{TextColor}{HTML}{333333}
\definecolor{DarkBackground}{HTML}{C92C3C}
\definecolor{LightBackground}{HTML}{FBF1F2}
\renewcommand{\familydefault}{\sfdefault}
\color{TextColor}

% Header and Footer
\pagestyle{fancy}
\fancyhead{} % Set header to blank
\fancyfoot{} % Set footer to blank
\fancyhead[L]{
\noindent
\begin{multicols}{3}
\begin{tabulary}{5.8cm}{C}
    \SetRowColor{DarkBackground}
    \vspace{-7pt}
    {\parbox{\dimexpr\textwidth-2\fboxsep\relax}{\noindent
        \hspace*{-6pt}\includegraphics[width=5.8cm]{/web/www.cheatography.com/public/images/cheatography_logo.pdf}}
    }
\end{tabulary}
\columnbreak
\begin{tabulary}{11cm}{L}
    \vspace{-2pt}\large{\bf{\textcolor{DarkBackground}{\textrm{C++ Pointers cookbook Cheat Sheet}}}} \\
    \normalsize{by \textcolor{DarkBackground}{Nima (nimakarimian)} via \textcolor{DarkBackground}{\uline{cheatography.com/113429/cs/21694/}}}
\end{tabulary}
\end{multicols}}

\fancyfoot[L]{ \footnotesize
\noindent
\begin{multicols}{3}
\begin{tabulary}{5.8cm}{LL}
  \SetRowColor{FootBackground}
  \mymulticolumn{2}{p{5.377cm}}{\bf\textcolor{white}{Cheatographer}}  \\
  \vspace{-2pt}Nima (nimakarimian) \\
  \uline{cheatography.com/nimakarimian} \\
  \end{tabulary}
\vfill
\columnbreak
\begin{tabulary}{5.8cm}{L}
  \SetRowColor{FootBackground}
  \mymulticolumn{1}{p{5.377cm}}{\bf\textcolor{white}{Cheat Sheet}}  \\
   \vspace{-2pt}Published 29th January, 2020.\\
   Updated 29th January, 2020.\\
   Page {\thepage} of \pageref{LastPage}.
\end{tabulary}
\vfill
\columnbreak
\begin{tabulary}{5.8cm}{L}
  \SetRowColor{FootBackground}
  \mymulticolumn{1}{p{5.377cm}}{\bf\textcolor{white}{Sponsor}}  \\
  \SetRowColor{white}
  \vspace{-5pt}
  %\includegraphics[width=48px,height=48px]{dave.jpeg}
  Measure your website readability!\\
  www.readability-score.com
\end{tabulary}
\end{multicols}}




\begin{document}
\raggedright
\raggedcolumns

% Set font size to small. Switch to any value
% from this page to resize cheat sheet text:
% www.emerson.emory.edu/services/latex/latex_169.html
\footnotesize % Small font.

\begin{multicols*}{4}

\begin{tabularx}{3.833cm}{X}
\SetRowColor{DarkBackground}
\mymulticolumn{1}{x{3.833cm}}{\bf\textcolor{white}{Array of Pointers}}  \tn
% Row 0
\SetRowColor{LightBackground}
\mymulticolumn{1}{x{3.833cm}}{int *ptr{[}arraysize{]} = array of pointers, pointing to int} \tn 
% Row Count 2 (+ 2)
\hhline{>{\arrayrulecolor{DarkBackground}}-}
\end{tabularx}
\par\addvspace{1.3em}

\begin{tabularx}{3.833cm}{X}
\SetRowColor{DarkBackground}
\mymulticolumn{1}{x{3.833cm}}{\bf\textcolor{white}{Pointer to function}}  \tn
\SetRowColor{LightBackground}
\mymulticolumn{1}{x{3.833cm}}{int ({\emph{FuncPTR)(int a,int b); //called funcptr is a pointer to function \newline // actually is used as a parameter of another func and can pass any func to the desired func as a parameter with this method \newline int func1(int); \newline int func2(int); \newline int func3(int (}}FuncPTR)(int)); \newline now we can pass func1 or func2 to func3 ;} \tn 
\hhline{>{\arrayrulecolor{DarkBackground}}-}
\end{tabularx}
\par\addvspace{1.3em}

\begin{tabularx}{3.833cm}{X}
\SetRowColor{DarkBackground}
\mymulticolumn{1}{x{3.833cm}}{\bf\textcolor{white}{Array of Pointers to functions}}  \tn
\SetRowColor{LightBackground}
\mymulticolumn{1}{x{3.833cm}}{void (*f{[}3{]})(int)=\{function1,function2,function3\}; \newline // f is an array of pointers , pointing to functions of type void which all of them take one parameter of type int .} \tn 
\hhline{>{\arrayrulecolor{DarkBackground}}-}
\end{tabularx}
\par\addvspace{1.3em}

\begin{tabularx}{3.833cm}{X}
\SetRowColor{DarkBackground}
\mymulticolumn{1}{x{3.833cm}}{\bf\textcolor{white}{Pointer vs array}}  \tn
\SetRowColor{LightBackground}
\mymulticolumn{1}{x{3.833cm}}{int b{[}10{]}; int {\emph{bptr; \newline bptr=b; OR bptr=\&b{[}0{]}; \newline }}(bptr+3) // shows the value of b{[}3{]} \newline  bptr+3 // points to \&b{[}3{]}  \newline //an array can be used like a pointer too -\textgreater{} *(b+3)=value of b{[}3{]}  \newline //pointer to an array can be used like an array -\textgreater{} bptr{[}3{]} = value of b{[}3{]}} \tn 
\hhline{>{\arrayrulecolor{DarkBackground}}-}
\end{tabularx}
\par\addvspace{1.3em}

\begin{tabularx}{3.833cm}{X}
\SetRowColor{DarkBackground}
\mymulticolumn{1}{x{3.833cm}}{\bf\textcolor{white}{NULL pointer}}  \tn
\SetRowColor{LightBackground}
\mymulticolumn{1}{x{3.833cm}}{int  *ptr = NULL; //The value of ptr is 0 \newline if(ptr)     // succeeds if p is not null \newline if(!ptr)    // succeeds if p is null} \tn 
\hhline{>{\arrayrulecolor{DarkBackground}}-}
\end{tabularx}
\par\addvspace{1.3em}

\begin{tabularx}{3.833cm}{X}
\SetRowColor{DarkBackground}
\mymulticolumn{1}{x{3.833cm}}{\bf\textcolor{white}{passing pointers to functions}}  \tn
\SetRowColor{LightBackground}
\mymulticolumn{1}{x{3.833cm}}{{\bf{{\emph{WHEN PASSING ARGUMENTS}}}} \newline    unsigned long sec;        \newline    getSeconds( \&sec ); \newline {\bf{{\emph{IN FUNCTION HEADER}}}} \newline void getSeconds(unsigned long *par)} \tn 
\hhline{>{\arrayrulecolor{DarkBackground}}-}
\end{tabularx}
\par\addvspace{1.3em}

\begin{tabularx}{3.833cm}{X}
\SetRowColor{DarkBackground}
\mymulticolumn{1}{x{3.833cm}}{\bf\textcolor{white}{Pointer to pointer}}  \tn
\SetRowColor{LightBackground}
\mymulticolumn{1}{x{3.833cm}}{A pointer to a pointer is a form of multiple indirection or a chain of pointers  \newline    int  var; \newline    int  {\emph{ptr; \newline    int  }}*pptr; \newline    var = 3000; \newline    ptr = \&var; // take the address of var \newline    pptr = \&ptr;// take the address of ptr using address of operator \&} \tn 
\hhline{>{\arrayrulecolor{DarkBackground}}-}
\end{tabularx}
\par\addvspace{1.3em}

\begin{tabularx}{3.833cm}{X}
\SetRowColor{DarkBackground}
\mymulticolumn{1}{x{3.833cm}}{\bf\textcolor{white}{Return pointer from functions}}  \tn
\SetRowColor{LightBackground}
\mymulticolumn{1}{x{3.833cm}}{int {\emph{ getRandom( ) \{ \newline    static int  r{[}10{]}; \newline    return r; \newline \} \newline // main function to call above defined function. \newline int main () \{ \newline    int }}p; \newline    p = getRandom(); \newline    \}} \tn 
\hhline{>{\arrayrulecolor{DarkBackground}}-}
\end{tabularx}
\par\addvspace{1.3em}

\begin{tabularx}{3.833cm}{X}
\SetRowColor{DarkBackground}
\mymulticolumn{1}{x{3.833cm}}{\bf\textcolor{white}{Array of strings}}  \tn
\SetRowColor{LightBackground}
\mymulticolumn{1}{x{3.833cm}}{char *color=\{"red","blue","yellow','green"\};  \newline //color is an array of pointers , pointing to the first character of each string} \tn 
\hhline{>{\arrayrulecolor{DarkBackground}}-}
\end{tabularx}
\par\addvspace{1.3em}


% That's all folks
\end{multicols*}

\end{document}
